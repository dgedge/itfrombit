% !TEX program = pdflatex
\documentclass[12pt,a4paper]{article}
\usepackage[utf8]{inputenc}
\usepackage[T1]{fontenc}
\usepackage{amsmath,amssymb,amsthm}
\usepackage{graphicx}
\usepackage{booktabs}
\usepackage{hyperref}
\usepackage[margin=2.5cm]{geometry}
\usepackage{natbib}
\bibliographystyle{unsrtnat}

\newcommand{\half}{\tfrac{1}{2}}
\newcommand{\nuR}{\nu_R}

\newtheorem{conjecture}{Conjecture}
\newtheorem{result}{Result}

\title{The Circlette Lattice\\
Background Independence, Fermion Doubling, and the\\
Topological Structure of the Holographic Vacuum\\[6pt]
\large Sequel to: ``The Holographic Circlette'' (Part~I)}
\author{D.G.\ Elliman\footnote{Email: dave@neusym.ai}\\
\textit{Neuro-Symbolic Ltd, United Kingdom}}
\date{February 2026 (v2)}

\begin{document}
\maketitle

\begin{center}
\textit{The lattice does not simulate quantum mechanics. It is quantum mechanics.}
\end{center}

\begin{abstract}
In the companion paper (Part~I)~\cite{Elliman2026}, we showed that the
Standard Model fermion spectrum emerges as the set of valid codewords of an
8-bit error-correcting code (the circlette) on a 2D holographic lattice, and
derived the Dirac and Schr\"odinger equations as the continuum limit of a
CNOT quantum walk.  Part~I~(v2) further derived the charged lepton mass
ratios to 0.007\% precision from a single geometric parameter
$\delta = 2/9$ - the ratio of a 2-bit topological defect to a 9-qubit
plaquette - together with the weak mixing angle $\sin^2\theta_W = 2/9$ and
the flavour mixing angles.

Here we address the foundational question: \emph{what is the lattice?}

We propose that the circlettes are not entities on a lattice but rather
\emph{constitute} the lattice itself - a background-independent quantum
cellular automaton in which spacetime is an emergent property of
informational adjacency.  We identify the 4.8.8 truncated square tiling as
the natural topological graph, with octagons as self-contained 8-bit
registers (matter) and interstitial squares as communication channels (gauge
links).

The 9-qubit plaquette - 8~boundary bits on the octagon ring plus 1~bulk
parity bit at the centre - provides the geometric origin of the key
parameter $\delta = d/N = 2/9$, where $d = 2$ is the support of the $\nuR$
topological defect and $N = 9$ is the plaquette size.  This integer ratio
determines the charged lepton mass spectrum (via the Koide
formula~\cite{Koide1983}), the weak mixing angle, and the flavour mixing
angles.  The colour dilution pattern $\delta_u = 2/27$, $\delta_d = 1/9$
for the quark sectors follows from the interaction of the defect with the
$N_c = 3$ colour multiplicity.

We investigate whether the 4.8.8 topology resolves the Nielsen-Ninomiya
fermion doubling problem~\cite{NielsenNinomiya1981a,NielsenNinomiya1981b}.
By explicit construction of the tight-binding Dirac operator, we prove that
the physical next-nearest-neighbour coupling through interstitial squares
generates a term proportional to $\alpha_1\alpha_2 \times \sin k_x \sin
k_y$, which vanishes at all doubler locations and cannot gap the unphysical
species.  However, the circlette's CNOT coin operator - the discrete $Z_2$
chiral update - provides a dynamical resolution: it breaks the continuous
$\text{U}(1)_A$ symmetry that the Nielsen-Ninomiya theorem requires,
rendering the theorem inapplicable at the axiomatic level.

This establishes a clean separation of concerns: the \emph{hardware} (4.8.8
topology) provides spatial structure, bandwidth matching, gauge plaquettes,
and the 9-qubit geometry from which $\delta = 2/9$ emerges; the
\emph{software} (CNOT dynamics) provides particle physics, chiral mixing,
mass generation, and doubler resolution.
\end{abstract}

\noindent\textbf{Keywords:} background independence, quantum cellular
automaton, fermion doubling, Nielsen-Ninomiya theorem, truncated square
tiling, lattice gauge theory, CNOT gate, discrete chirality, 9-qubit
plaquette

\tableofcontents
\newpage

%======================================================================
\section{Introduction}
%======================================================================

In Part~I of this series~\cite{Elliman2026}, we established that 45 Standard
Model fermion states emerge as the valid codewords of an 8-bit ring code on a
holographic lattice, and that the Dirac equation arises as the continuum
limit of a CNOT quantum walk on the chirality--isospin bits.  Part~I~(v2)
extended the framework to derive, from the single geometric parameter
$\delta = 2/9$:
\begin{itemize}
\item the charged lepton mass ratios to 0.007\% precision,
\item the weak mixing angle $\sin^2\theta_W = 2/9$ (0.5\% error),
\item the $W/Z$ boson mass ratio $M_W/M_Z = \sqrt{7/9}$ (0.06\% error),
\item the Cabibbo angle and PMNS neutrino mixing angles (2--5\% errors), and
\item the colour dilution pattern for quark masses ($\delta_u = 2/27$,
  $\delta_d = 1/9$).
\end{itemize}

But a foundational question was deferred: \emph{what is the lattice itself?}
In Part~I, the lattice was treated as a given substrate - a background graph
on which the circlettes reside.  This is unsatisfactory for two reasons.
First, it introduces an unexplained scaffold; second, it fails to address a
standard objection from lattice gauge theory: the Nielsen-Ninomiya fermion
doubling problem~\cite{NielsenNinomiya1981a,NielsenNinomiya1981b}.

In this paper we address both issues.  Section~\ref{sec:ontology} establishes
the ontological framework.  Section~\ref{sec:488} identifies the 4.8.8
tiling and derives the bandwidth-matching principle.  Section~\ref{sec:plaq}
shows how the 9-qubit plaquette geometry gives rise to $\delta = 2/9$ and
the mass spectrum.  Section~\ref{sec:lgt} establishes the correspondence
with lattice gauge theory.  Section~\ref{sec:doubling} contains the central
technical result on fermion doubling.  Section~\ref{sec:arch} discusses the
hardware/software separation.  Section~\ref{sec:cosmo} sketches cosmological
implications.


%======================================================================
\section{Ontology: Background Independence and the Vacuum}
\label{sec:ontology}
%======================================================================

\subsection{Three Pictures of the Lattice}

Consider three possible relationships between circlettes and the lattice:

\begin{description}
\item[Picture A (Tenant Model):] The lattice is a separate graph; circlettes
  are states at its nodes.  This is the working assumption of Part~I.
\item[Picture B (Identity Model):] The circlettes \emph{are} the nodes.
  ``Space'' is the graph of who-connects-to-whom.  No separate lattice
  exists.
\item[Picture C (Woven Model):] The circlettes share edges with their
  neighbours; the lattice is woven from overlapping rings.
\end{description}

We adopt Picture~B as the ontological foundation.  Picture~A requires
explaining what the lattice is made of (substrate regress).  Picture~C
introduces shared degrees of freedom that conflict with the self-contained
bit ownership of the code.  Picture~B is background-independent: it defines
spatial relationships purely through informational adjacency, with no
embedding space required.

\subsection{Architecture}

The fundamental objects are:
\begin{description}
\item[Circlette:] An isolated 8-bit register.  Fully self-contained; owns
  all its bits absolutely.  Has an internal state that is either a valid
  codeword (one of~45) or an error state.
\item[Link:] A communication channel between two circlettes.  Carries a
  U(1) phase (the gauge field).
\item[Lattice:] The graph of circlettes connected by links.  Not a separate
  entity - defined entirely by which circlettes are linked to which.
\end{description}

This is precisely the architecture of Hamiltonian lattice gauge theory in the
Kogut-Susskind formulation~\cite{KogutSusskind1975}: matter fields live at
sites, gauge fields live on links.

\subsection{The Vacuum Is Not Empty}

Every node in the lattice is occupied by a circlette in its ground state.
``Empty space'' is a lattice of circlettes all in the vacuum configuration.
Removing the circlettes does not produce empty space - it produces nothing:
no distance, no dimension, no geometry.  The lattice is load-bearing.

\subsection{Particles as Information Gliders}

Particle propagation is information flow across a stationary substrate.  The
circlettes do not move; the pattern moves.  At each tick, the CNOT quantum
walk transfers the excitation from one circlette to its neighbour through the
communication channel, evaluating the U(1) link phase in
transit~\cite{Gardner1970}.


%======================================================================
\section{The 4.8.8 Topology}
\label{sec:488}
%======================================================================

\subsection{The Truncated Square Tiling}

We identify the topological graph of the circlette lattice with the 4.8.8
Archimedean tiling (the truncated square tiling), in which regular octagons
and squares tile the plane with vertex configuration $(4 \cdot 8^2)$.

\subsection{The Bandwidth-Matching Principle}

Each octagon in the 4.8.8 tiling has exactly 8 edges - the same as the
number of bits in the circlette.  These partition into two geometric sectors:
\begin{enumerate}
\item \textbf{4~orthogonal channels} connecting to nearest-neighbour
  octagons (N, S, E, W).  These carry the Dirac kinetic term.
\item \textbf{4~diagonal channels} connecting through interstitial squares to
  next-nearest-neighbour octagons (NE, NW, SE, SW).  These provide the
  gauge plaquette structure.
\end{enumerate}
The total channel count is:
\begin{equation}
n_{\text{channels}} = 4_{\text{kinematic}} + 4_{\text{gauge}} = 8
= n_{\text{bits}}.
\end{equation}

\begin{conjecture}[Bandwidth Matching]
The 4.8.8 truncated square tiling is the unique Archimedean tiling of the
plane for which the polygon coordination number equals the minimum code
length required to support the Standard Model fermion spectrum.
\end{conjecture}

\subsection{Orthogonal Gauge Separation}

On a standard square lattice, the kinematic hops and gauge plaquettes share
the same edges.  On the 4.8.8 lattice, they are topologically distinct:
fermion hops use octagon-to-octagon edges, while gauge loops traverse the
interstitial square boundaries.  This provides a natural separation of data
from forces.

\begin{conjecture}[Single-Tick Consistency]
The 4.8.8 topology is the minimal planar graph that permits both the CNOT
kinematic update and the U(1) gauge evaluation within a single computational
tick without channel conflict.
\end{conjecture}


%======================================================================
\section{The 9-Qubit Plaquette and the Origin of $\delta = 2/9$}
\label{sec:plaq}
%======================================================================

This section, new to v2, establishes the connection between the 4.8.8
topology and the quantitative predictions of Part~I.

\subsection{The Plaquette as Unit Cell}

Each octagon of the 4.8.8 tiling is an 8-bit ring: the circlette.  But the
\emph{unit cell} of the tiling includes one additional degree of freedom:
the bulk parity (or syndrome) bit at the centre of the plaquette, shared
among the stabiliser constraints.  The total unit cell is therefore a
\textbf{9-qubit plaquette}:
\begin{itemize}
\item 8~boundary sites: the ring bits $G_0, G_1, C_0, C_1,
  \text{LQ}, I_3, \chi, W$,
\item 1~centre site: the bulk parity.
\end{itemize}
The vacuum state (ground state of the stabiliser Hamiltonian) is delocalised
across all 9~sites.

\subsection{The $\nuR$ Topological Defect}

Three pseudocodewords - one per generation - satisfy constraints R1--R3 but
violate only R4 (no right-handed neutrino).  These $\nuR$ states are
localised at the 2~boundary sites where the constraint is violated.

The defect has three key properties:
\begin{enumerate}
\item \textbf{Localisation:} It is pinned to $d = 2$ sites and cannot spread
  without additional energy cost.
\item \textbf{Three-fold degeneracy:} The $Z_3$ symmetry of the generation
  ring admits three $\nuR$ states.
\item \textbf{Boundary character:} It lives on the boundary of the
  plaquette, not in the bulk.
\end{enumerate}

\subsection{The Geometric Origin of $\delta$}

The vacuum translation amplitude scales with the full plaquette:
$T_{\text{vac}} \propto Nt$, where $N = 9$.  The defect, pinned to its
2-site support, has $T_{\text{def}} \propto dt$, where $d = 2$.  The
geometric (Berry) phase is:
\begin{equation}
\delta = \frac{d}{N} = \frac{2}{9}~\text{rad}
\label{eq:delta}
\end{equation}

This single ratio determines:
\begin{itemize}
\item The Koide mass formula phase:
  $m_n = \mu(1 + \sqrt{2}\cos(\delta + 2\pi n/3))^2$~\cite{Koide1983,PDG2024}.
\item The weak mixing angle: $\sin^2\theta_W = d/N = 2/9$.
\item The $W/Z$ boson mass ratio: $M_W/M_Z = \sqrt{(N-d)/N} = \sqrt{7/9}$.
\item The Cabibbo angle: $\theta_C \approx \delta$.
\end{itemize}

The 4.8.8 topology is no longer merely a convenient graph.  It is the
\emph{source} of the integer ratio $2/9$ that parameterises the Standard
Model.

\subsection{Colour Dilution and the Quark Sector}

Part~I~(v2) showed that the quark mass sectors follow the same Koide
functional form but with modified geometric parameters.  Fitting to
experimental quark masses~\cite{PDG2024} reveals:

\begin{table}[ht]
\centering
\begin{tabular}{lcccl}
\toprule
Sector & $\delta$ & $R$ & Source & Status \\
\midrule
Leptons & $2/9$ & $\sqrt{2}$ & Base geometry & Rigorous \\
Up quarks & $2/27$ & $\approx\!\sqrt{3}$ & Twist$/N_c$ & Structural \\
Down quarks & $1/9$ & $\approx\!1.55$ & Twist$/2$ & Phenomenological \\
\bottomrule
\end{tabular}
\caption{Geometric parameters for each charge sector.  Colour introduces a
dilution factor in the twist: the 2-bit defect is shared across $N_c = 3$
colour sheets for up quarks, and across 2~isospin channels for down quarks.}
\label{tab:colour}
\end{table}

The colour dilution hypothesis $\delta_u = \delta_\ell / N_c = 2/27$ is
accurate to 0.6\% against the fitted value.  The down quark twist
$\delta_d = \delta_\ell / 2 = 1/9$ is accurate to 1.1\%.  The down sector
reproduces $m_d$ and $m_s$ within experimental uncertainties (3.6\% and 1.0\%
respectively).  The up sector correctly predicts the heavy quark hierarchy
($m_c/m_t$) but the lightest mass $m_u$ sits near a node of the cosine
function and requires sub-leading corrections.

In the lattice picture, the physical mechanism is clear: the defect's Berry
phase is diluted because the colour degree of freedom provides $N_c$
parallel sheets through which the defect can propagate, each carrying a
fraction $1/N_c$ of the total phase.


%======================================================================
\section{Correspondence with Lattice Gauge Theory}
\label{sec:lgt}
%======================================================================

\subsection{Kogut-Susskind Architecture}

The circlette lattice maps directly onto the Kogut-Susskind formulation of
Hamiltonian lattice gauge theory~\cite{KogutSusskind1975}:

\begin{table}[ht]
\centering
\begin{tabular}{lll}
\toprule
LGT concept & Circlette realisation & 4.8.8 element \\
\midrule
Site (matter field) & 8-bit register, 45-state code & Octagon \\
Link (gauge field) & U(1) phase $U(x,y) = e^{ieA\Delta x}$ &
  Oct--oct edge \\
Plaquette (field strength) & Wilson loop $\prod_\square U$ &
  Interstitial square \\
Hopping term & CNOT quantum walk & NN channel \\
Wilson mass & CNOT coin operator & Built into dynamics \\
\bottomrule
\end{tabular}
\caption{Mapping between lattice gauge theory and the circlette lattice.}
\end{table}

\subsection{Wilson Loops and Plaquettes}

The interstitial squares of the 4.8.8 tiling are 4-cycles.  In lattice gauge
theory, the phase accumulated around the shortest closed loop is the
plaquette variable~\cite{Wilson1974}:
\begin{equation}
U_\square = U_{12}\,U_{23}\,U_{34}\,U_{41},
\end{equation}
which gives the discretised field strength tensor $F_{\mu\nu}$ in the
continuum limit.

\subsection{Anomaly Structure from the Code}

As shown in Part~I, the 45-state code automatically satisfies:
\begin{align}
\sum_f Q_f &= 0 \quad\text{(gravitational anomaly cancellation)}, \\
\sum_f Q_f^2 &= 16 \quad\text{(exact Standard Model $\beta$-function
  coefficient)}.
\end{align}


%======================================================================
\section{The Fermion Doubling Problem}
\label{sec:doubling}
%======================================================================

\subsection{Background: the Nielsen-Ninomiya Theorem}

The Nielsen-Ninomiya no-go
theorem~\cite{NielsenNinomiya1981a,NielsenNinomiya1981b} states that on any
regular lattice, a Dirac operator satisfying locality, translational
invariance, Hermiticity, and exact continuous $\text{U}(1)_A$ chiral symmetry
necessarily has an equal number of left-handed and right-handed fermion
species.  In 2D, this produces 4~Dirac points (1 physical, 3 doublers).

The standard remedy is the Wilson mass~\cite{Wilson1977}: a
momentum-dependent term $M(k) = r(2 - \cos k_x - \cos k_y)$ proportional to
$\beta$ that gaps the doublers while leaving the physical fermion massless.
This is added by hand and explicitly breaks chiral symmetry.

\subsection{The Dirac Operator on the 4.8.8 Graph}

In the circlette framework, the coin space is spanned by the chirality bit
$\chi$ and isospin bit $I_3$, giving 4-component spinors.  The Dirac
matrices are:
\begin{equation}
\alpha_1 = \sigma_x^\chi \otimes \sigma_x^{I_3}, \quad
\alpha_2 = \sigma_x^\chi \otimes \sigma_y^{I_3}, \quad
\beta = \sigma_z^\chi \otimes I^{I_3}.
\end{equation}
On the square lattice with nearest-neighbour (NN) hopping only, the naive
Bloch Hamiltonian is:
\begin{equation}
H_{\text{naive}}(k) = \alpha_1 \sin k_x + \alpha_2 \sin k_y,
\end{equation}
which has zeros at $(0,0)$, $(\pi,0)$, $(0,\pi)$, and $(\pi,\pi)$ - the
standard doubling problem.

\subsection{The Geometric Temptation: NNN Hopping}

The 4.8.8 topology provides next-nearest-neighbour (NNN) connections through
the interstitial squares.  A natural hypothesis is that these diagonal
channels act as a native Wilson mass.

The NNN coupling matrix from the two-hop process gives:
\begin{equation}
H_{\text{NNN}}(k) = 4t_2\, \sin k_x \sin k_y \cdot \alpha_1\alpha_2,
\end{equation}
where:
\begin{equation}
\alpha_1\alpha_2 = (\sigma_x \otimes \sigma_x)(\sigma_x \otimes \sigma_y)
= I \otimes (i\sigma_z) = i(I^\chi \otimes \sigma_z^{I_3}).
\end{equation}
This is a rotation in the isospin sector, \emph{not} the mass matrix $\beta$
which acts on chirality.

\subsection{The Rigorous Disproof}

The full 4.8.8 Bloch Hamiltonian is:
\begin{equation}
H_{4.8.8}(k) = \underbrace{\alpha_1 \sin k_x + \alpha_2 \sin k_y}_{\text{NN
(Dirac)}} + \underbrace{4t_2\, \sin k_x \sin k_y\, \alpha_1\alpha_2}_{
\text{NNN (diagonal)}}.
\end{equation}

At all three doubler locations:
\begin{align}
H_{4.8.8}(\pi, 0) &: \sin\pi\cdot\sin 0 = 0 \Rightarrow
\text{NNN term vanishes}, \\
H_{4.8.8}(0, \pi) &: \sin 0\cdot\sin\pi = 0 \Rightarrow
\text{NNN term vanishes}, \\
H_{4.8.8}(\pi, \pi) &: \sin\pi\cdot\sin\pi = 0 \Rightarrow
\text{NNN term vanishes}.
\end{align}

\begin{result}
The physical NNN coupling generated by the 4.8.8 topology vanishes at all
three doubler locations.  The 4.8.8 geometry alone cannot resolve the fermion
doubling problem, regardless of the NNN coupling strength~$t_2$.
\end{result}

This was confirmed by numerical diagonalisation across the full Brillouin
zone~\cite{FermionDoublingCode}.  Furthermore, exhaustive testing of all
possible NNN coupling matrices with a Wilson-like $\cos k_x\cos k_y$
dependence showed that the $(\pi,\pi)$ doubler cannot be distinguished from
the physical fermion at $(0,0)$ because $\cos\pi\cos\pi = \cos 0\cos 0 = 1$.

\subsection{The Algorithmic Resolution: the CNOT Coin}

The resolution comes not from the spatial topology but from the dynamics.

In the theory of discrete-time quantum walks~\cite{DAriano2014,Bisio2015,
Venegas2012}, the coin operator is the sole generator of the effective mass
in the continuum limit.  In the circlette framework, the coin operator is the
CNOT gate:
\begin{equation}
\text{CNOT}: \quad \chi \to \chi, \quad W \to W \oplus \chi.
\end{equation}
This copies the chirality bit into the weak bit, enforcing constraint R2 at
each tick - an explicit, dynamical mixing of left- and right-handed chiral
sectors at every computational step.

The Nielsen-Ninomiya theorem is a topological winding theorem: it requires
that the continuous phase $\text{U}(1)_A$ accumulated around the Brillouin
zone torus sums to zero.  But in the circlette framework:
\begin{enumerate}
\item Chirality is a discrete Boolean variable $\chi \in \{0,1\}$, not a
  continuous phase.
\item The ``chiral symmetry'' is $Z_2$ (bit flip), not $\text{U}(1)_A$.
\item A discrete $Z_2$ space does not support topological winding.
  Formally, $\pi_1(\text{U}(1)) = \mathbb{Z}$ permits integer winding
  numbers; a discrete $Z_2$ space has no continuous manifold, and the
  homotopy prerequisites are absent.
\end{enumerate}

\begin{result}
The Nielsen-Ninomiya theorem is inapplicable to the circlette lattice
because its foundational premise - exact continuous $\text{U}(1)_A$ chiral
symmetry - is violated by the discrete $Z_2$ structure of the chirality bit
and its CNOT update rule.
\end{result}

The doublers are not gapped by a penalty term; they never arise because the
symmetry that would protect them does not exist in the discrete theory.

\begin{table}[ht]
\centering
\begin{tabular}{lcc}
\toprule
& Wilson fermions & Circlette (CNOT coin) \\
\midrule
Chiral symmetry & $\text{U}(1)_A$ broken explicitly &
  $Z_2$ (never continuous) \\
Mechanism & Momentum-dependent mass & Discrete chiral mixing \\
Origin & Added by hand & Fundamental dynamics \\
Level & Hamiltonian (spatial) & Coin operator (temporal) \\
Nielsen-Ninomiya & Evaded (symmetry broken) & Inapplicable (wrong axiom) \\
Physical mass & From Wilson parameter $r$ & From CNOT frequency \\
\bottomrule
\end{tabular}
\caption{Comparison of doubler resolution mechanisms.}
\end{table}


%======================================================================
\section{Hardware and Software: A Separation of Concerns}
\label{sec:arch}
%======================================================================

\subsection{The Architecture}

The results of the preceding sections establish a clean decoupling:

\paragraph{Hardware (the 4.8.8 topology):}
\begin{itemize}
\item Establishes the $n = 8$ bandwidth matching (code length = coordination
  capacity).
\item Separates kinematic channels from gauge channels.
\item Provides native Wilson-loop plaquettes as interstitial squares.
\item Defines the 9-qubit plaquette geometry from which $\delta = 2/9$
  emerges.
\item Sets the spatial adjacency structure from which distances emerge.
\end{itemize}

\paragraph{Software (the CNOT dynamics):}
\begin{itemize}
\item Generates the Dirac equation as the continuum limit of the quantum
  walk.
\item Provides dynamical chiral mixing, resolving fermion doubling.
\item Produces rest mass as the CNOT execution frequency.
\item Enforces the R1--R4 constraints that define the particle spectrum.
\item Generates the mass hierarchy through Feshbach resonance via the $\nuR$
  defect.
\end{itemize}

The hardware is the routing network; the software is the physics.  The
spatial topology determines who can talk to whom; the update rule determines
what they say.

\subsection{The Complete Causal Chain}

From the 4.8.8 topology to the Standard Model:

\begin{center}
\begin{tabular}{lcl}
4.8.8 tiling & $\to$ & $n_{\text{bits}} = 8$ (bandwidth matching) \\
8-bit ring + R1--R4 & $\to$ & 45 fermion states \\
Octagon + centre & $\to$ & 9-qubit plaquette \\
$\nuR$ defect on 2 sites & $\to$ & $\delta = d/N = 2/9$ \\
$Z_3$ circulant + quadrature & $\to$ & $m_n = \mu(1+\sqrt{2}\cos(\delta +
  2\pi n/3))^2$ \\
Bulk/boundary partition & $\to$ & $\sin^2\theta_W = 2/9$,
  $M_W/M_Z = \sqrt{7/9}$ \\
Colour sheets & $\to$ & $\delta_u = 2/27$, $\delta_d = 1/9$ \\
CNOT coin & $\to$ & Dirac equation, mass, doubler resolution \\
\end{tabular}
\end{center}

Every quantitative prediction of the framework traces back to the
topological properties of the 4.8.8 graph.

\subsection{Distance and Geometry}

In the background-independent picture, distance between two circlettes is the
number of hops (graph distance).  The link variables $U(x,y)$ modulate this:
regions with large phase fluctuations have shorter effective distances.  The
metric tensor emerges from the Fisher information geometry of the link
variables.  Curved spacetime is a region where the link variables vary
systematically.


%======================================================================
\section{Cosmological Implications}
\label{sec:cosmo}
%======================================================================

\subsection{The Past Hypothesis and the Origin of Time}

Time is emergent from the sequential execution of the CNOT update rule.
There is no external clock; the tick $t \to t+1$ \emph{is} time.  Asking
``what happened before the lattice?'' is a category error~\cite{Wheeler1990}.

The Big Bang is a symmetry-breaking phase transition of a pre-geometric
lattice.  The initial state is maximally symmetric (every node in the same
ground state), giving Shannon entropy $S = 0$: the universe began at minimum
entropy not because it was empty, but because it was perfectly
uniform~\cite{Penrose1989}.

\subsection{Holographic Expansion}

The holographic principle~\cite{tHooft1993,Susskind1995} and the
Bekenstein-Hawking bound~\cite{Bekenstein1973} require $S_{\max} \propto
A/4\ell_P^2$.  A fixed graph would cap total entropy, violating the Second
Law.  This necessitates a growing lattice: cosmic expansion corresponds to
the continuous addition of new nodes to the 4.8.8 boundary graph.

The vacuum occupation fraction $\Phi = 45/256$ becomes the energy density per
node.  As the graph grows, the total vacuum energy increases proportionally
with the node count (hence with the volume), naturally producing the constant
energy density $\rho_\Lambda$ that drives accelerated expansion.

\subsection{Algorithmic Baryogenesis}

The CNOT gate is an inherently asymmetric, directed operation: the chirality
bit $\chi$ controls the update of the weak bit $W$, but $W$ does not alter
$\chi$.  This control-target hierarchy is not invariant under the combined
operation of CP.  Whether this structural non-commutativity, propagated
through the $Z_3$ generation sector, natively generates the CKM complex phase
remains an open question.


%======================================================================
\section{Open Questions}
%======================================================================

\subsection{Is the 4.8.8 Tiling Unique?}

The bandwidth-matching argument is currently heuristic.  A proof would
require showing that the 4.8.8 tiling is the unique Archimedean tiling with
coordination number~8 and bipartite gauge structure.  This is a well-posed
mathematical question: there are only 11 Archimedean tilings.

\subsection{Can the CNOT Resolution Be Made Rigorous?}

A rigorous proof would require explicit construction of the DTQW propagator
with the CNOT coin, computation of the effective continuum Lagrangian, and
demonstration that no doubler poles appear in the fermion
propagator~\cite{DAriano2014,Bisio2015,BialynickiBirula1994}.

\subsection{The Brillouin Zone Folding Question}

The bipartite structure of the 4.8.8 tiling doubles the real-space unit cell
and halves the Brillouin zone.  Whether the resulting folding absorbs the
final doubler into internal spinor degrees of freedom (as in Kogut-Susskind
staggered fermions~\cite{KogutSusskind1975}) remains an open question.

\subsection{The Down Quark Twist Factor}

The colour dilution pattern gives $\delta_u = \delta_\ell / N_c$ for up
quarks (0.6\% from the fit) and $\delta_d = \delta_\ell / 2$ for down quarks
(1.1\% from the fit).  The factor of~2 for the down sector may relate to
hypercharge or the isospin-doublet structure.  A first-principles derivation
from the $(C_0, C_1)$ colour bits is needed.

\subsection{The Down Quark Structure Factor}

The down quark structure factor $R_d \approx 1.546$ does not correspond to a
simple integer root ($R_d^2 \approx 2.39$).  The closest candidate is
$\sqrt{12/5}$ (0.2\% error), but the geometric interpretation remains
unclear.


%======================================================================
\section{Conclusions}
%======================================================================

We have established the ontological and topological foundations of the
circlette lattice:

\begin{enumerate}
\item \textbf{Background independence:} The circlettes \emph{are} the
  lattice.  Spacetime is an emergent property of informational adjacency.

\item \textbf{The 4.8.8 topology:} Provides the natural graph with
  $n_{\text{bits}} = n_{\text{channels}} = 8$ (bandwidth matching),
  separated kinematic and gauge channels, and native Wilson-loop plaquettes.

\item \textbf{The 9-qubit plaquette:} The unit cell of the 4.8.8 tiling
  (8~boundary $+$ 1~centre) gives $\delta = d/N = 2/9$ - the single integer
  ratio that parameterises the charged lepton mass spectrum, the weak mixing
  angle, and the flavour mixing angles.

\item \textbf{Colour dilution:} The defect Berry phase is divided by the
  colour multiplicity for quarks: $\delta_u = 2/27$ (twist$/N_c$),
  $\delta_d = 1/9$ (twist$/2$).

\item \textbf{Fermion doubling:} The NNN coupling from the 4.8.8 topology
  vanishes at all doubler locations (rigorously verified).  The CNOT coin
  operator's $Z_2$ chiral mixing makes the Nielsen-Ninomiya theorem
  inapplicable.

\item \textbf{Hardware/software separation:} The topology provides spatial
  structure and the geometric ratio $2/9$; the dynamics provides particle
  physics and doubler resolution.
\end{enumerate}

The deepest result of Parts~I and~II combined is that every quantitative
prediction of the framework - the mass ratios, the mixing angles, the gauge
couplings - traces back through the CNOT dynamics to the integer topology of
the 4.8.8 graph.  The lattice does not simulate quantum mechanics.  It is
quantum mechanics.


%======================================================================
\bibliography{circlette-lattice-refs}
%======================================================================

\end{document}
